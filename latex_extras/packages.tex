%These are the packages that we will use in our latex document. 

\usepackage{graphicx}
\usepackage{natbib}	%Bibliography package. It handles citations and can easily change formats. This is done at the end of the document
\usepackage[utf8]{inputenx}% For proper input encoding
\usepackage{adjustbox} %Handles resizing of tables, figures, etc. Based off of page and/or line width/height 
% Packages for tables
\usepackage{booktabs}% For Pretty tables
\usepackage{threeparttable}% For Notes below table
\usepackage{rotating}% To Rotate Table
\usepackage{amsmath, amssymb,mathrsfs} %For using math
\usepackage{bm}
\usepackage{caption} 
\usepackage[list=true]{subcaption} %For having multiple figures within the same one. Figure 1, with part (a) and (b)
%\setcounter{lofdepth}{2}
\usepackage{setspace} %Single, Double Space, etc 
\usepackage[paperwidth=8.5in, paperheight=11in,margin=1in]{geometry} %For controlling the dimensions. Very useful for Posters, etc 
\usepackage{chngpage} 
\usepackage{everypage}%For page numbers on rotated pages
\usepackage[capposition=top]{floatrow}
\usepackage{accents}
\usepackage{float}
\usepackage{comment}
\usepackage{morefloats}
\usepackage{placeins}% To Create Float Barriers so the tables will stay in their sections
\usepackage{pdflscape}

\usepackage{siunitx} %This aligns tables by their decimal and handles processing of the numbers within the tables
  \sisetup{
    detect-mode,
    group-digits      = false,
    input-symbols     = {( ) [ ] - +},
    table-align-text-post = false,
    input-signs             = ,
    %parse-numbers=false,
    %scientific-notation = true,
        %round-mode              = places,
        %round-precision         = 2,
        %input-ignore={,},
    %input-decimal-markers={.},
    %group-separator={,},
        } 
\usepackage{grffile}
\usepackage{soul}
\usepackage{color}
\usepackage{caption}
%\captionsetup[figure]{justification=raggedright,singlelinecheck=off}
\sisetup{separate-uncertainty=true}
\usepackage{mathptmx}
\DeclareCaptionLabelFormat{blank}{}
